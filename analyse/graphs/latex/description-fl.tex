\begin{longtable}{ll}
\caption{Descriptions des FL ciblées} \label{tab:descriptions-fl} \\
\toprule
node & description \\
\midrule
\endfirsthead
\caption[]{Descriptions des FL ciblées} \\
\toprule
node & description \\
\midrule
\endhead
\midrule
\multicolumn{2}{r}{Continued on next page} \\
\midrule
\endfoot
\bottomrule
\endlastfoot
FL\_paradigmatiques & Les FL\_paradigmatiques sont une classe de fonctions lexicales qui prennent un mot-clé comme argument, et renvoient les dérivations syntaxiques ou sémantiques comme valeur de ce mot-clé. \\
Substitutive & Les fonctions lexicales de substitutive sont une classe de fonctions lexicales qui, pour une lexie donnée, retournent une lexie de la même partie du discours ou une expression équivalente à cette partie du discours. Elles expriment la synonymie (exacte, approximative ou générique), ou une opposition de sens (antonymes, contrastifs, conversifs, etc.). \\
Subst\_sens\_similaire & Les fonctions lexicales de substitutive similaire de sens similaire sont une classe de fonctions lexicales qui, étant donné une lexie comme mot-clé, renvoient une autre lexie désignant une même entité ou un concept sémantiquement proche. \\
Syn & Syn est une fonction lexicale qui, étant donné une lexie comme mot-clé, renvoie une autre lexie désignant son synonyme exact ou approximatif. \\
Conv-ij & Conv-ij est une famille de fonctions lexicales qui, étant donné une lexie comme mot-clé, renvoie une autre lexie exprimant le même sens, mais avec une conversion des arguments \$i et \$j sur le plan sémantique. \\
Gener & Gener est une fonction lexicale qui, étant donné une lexie comme mot-clé, renvoie une autre lexie qui désignent son hyperonyme ou son terme générique, c'est-à-dire un terme plus général qui le catégorise. \\
Subst\_sens\_opposé & Les fonctions lexicales de substitutive opposé sont une classe de fonctions lexicales qui, étant donné une lexie comme mot-clé, renvoient une autre lexie comme valeur, désignant un antonyme, un contraste conceptuel ou simplement une conversion des arguments sémantiques. \\
Anti & Anti est une fonction lexicale qui, étant donné une lexie comme mot-clé, renvoie une autre lexie désignant son antonyme. \\
Contr & Contr est une fonction lexicale qui, étant donné une lexie comme mot-clé, renvoie une autre lexie désignant son contraste sémantique ou conceptuel, sans qu'il s'agisse nécessairement d'un antonyme \\
Dérivation\_nominale & Les fonctions lexicales de dérivation nominale sont une classe de fonctions lexicales qui, étant donné une lexie comme mot-clé, renvoient une autre lexie nominal sémantiquement dérivée de ce mot-clé. \\
S0 & S0 est une fonction lexicale de dérivation nominale vide de sens qui, étant donné une lexie comme mot-clé, renvoie une autre lexie nominal dérivée en dénotant le sens du mot-clé lui-même, sans ajouter de sens. \\
N\_dériv\_ajout\_de\_sens & N\_dériv\_ajout\_de\_sens est une classe de fonctions lexicales qui, étant donné une lexie comme mot-clé, renvoient une autre lexie nominale dérivée enrichie par un ajout de sens, qui dénote soit un argument sémantique soit un circonstant typique de ce mot-clé. \\
S\_i & S\_i est une classe de fonctions lexicales qui, étant donné une lexie comme mot-clé, renvoient une autre lexie nominale dérivée sémantiquement (pas forcément morphologiquement) qui dénote un certain argument sémantique (numéro \$i) de ce mot-clé. \\
S1 & S1 est une fonction lexicale qui, étant donné une lexie comme mot-clé, renvoie une autre lexie nominale dérivée sémantiquement (pas forcément morphologiquement) qui dénote le premier argument (\$1) du mot-clé. \\
S2 & S2 est une fonction lexicale qui, étant donné une lexie comme mot-clé, renvoie une autre lexie nominale dérivée sémantiquement (pas forcément morphologiquement) qui dénote le deuxième argument (\$2) du mot-clé. \\
S3 & S3 est une fonction lexicale qui, étant donné une lexie comme mot-clé, renvoie une autre lexie nominale dérivée sémantiquement (pas forcément morphologiquement) qui dénote le troisième argument (\$3) du mot-clé. \\
S4 & S4 est une fonction lexicale qui, étant donné une lexie comme mot-clé, renvoie une autre lexie nominale dérivée sémantiquement (pas forcément morphologiquement) qui dénote le quatrième argument (\$4) du mot-clé. \\
N\_dériv\_circonstancielle & Les fonctions lexicales de dérivation nominale circonstancielle sont une classe de fonctions lexicales qui, étant donné une lexie comme mot-clé, renvoie une autre lexie nominale dérivée sémantiquement (pas forcément morphologiquement) qui dénote un circonstant (instrument, méthode, manière, lieu ou résultat) typique du mot-clé. \\
Sinstr & Sinstr est une fonction lexicale qui, étant donné une lexie comme mot-clé, renvoie une autre lexie nominale dérivée sémantiquement (pas forcément morphologiquement) qui dénote un instrument typique de ce mot-clé. \\
Smed & Smed est une fonction lexicale qui, étant donné une lexie comme mot-clé, renvoie une autre lexie nominale dérivée sémantiquement (pas forcément morphologiquement) qui dénote un moyen ou une méthode typique de ce mot-clé. \\
Smod & Smod est une fonction lexicale qui, étant donné une lexie comme mot-clé, renvoie une autre lexie nominale dérivée sémantiquement (pas forcément morphologiquement) qui dénote une manière typique de ce mot-clé. \\
Sloc & Sloc est une fonction lexicale qui, étant donné une lexie comme mot-clé, renvoie une autre lexie nominale dérivée sémantiquement (pas forcément morphologiquement) qui dénote un lieu typique de ce mot-clé. \\
Sres & Sres est une fonction lexicale qui, étant donné une lexie comme mot-clé, renvoie une autre lexie nominale dérivée sémantiquement (pas forcément morphologiquement) qui dénote un résultat typique de ce mot-clé. \\
Dérivation\_adjectivale & Les fonctions lexicales de dérivation adjectivale sont une classe de fonctions lexicales qui, étant donné une lexie comme mot-clé, renvoient une lexie adjectivale dérivée tout en préservant un lien sémantique avec ce mot-clé. \\
A0 & A0 est une fonction lexicale qui, étant donné une lexie comme mot-clé, renvoi une autre lexie adjectivale dérivée qui est exactement de même sens que le mot-clé. \\
A\_dériv\_ajout\_de\_sens & Les fonctions lexicales de dérivation adjectivale vide de sens, sont une classe de fonctions lexicales qui, étant donné une lexie comme mot-clé, renvoient une autre lexie adjectivale dérivée qui caractérise l'argument \$i du mot-clé, soit par son propre sens, soit par le sens d'une capacité ou d'une possibilité. \\
A\_i & Les A\_i sont une classe de fonctions lexicales qui, étant donné une lexie comme mot-clé, renvoient une autre lexie adjectivale dérivée qui caractérise l'argument sémantique \$i du mot-clé par son sens. \\
A1 & A1 est une fonction lexicale qui, étant donné une lexie comme mot-clé, renvoie une autre lexie adjectivale dérivée qui caractérise le premier argument (\$1) du mot-clé par son sens. \\
A2 & A2 est une fonction lexicale qui, étant donné une lexie comme mot-clé, renvoie une autre lexie adjectivale dérivée qui caractérise le deuxième argument (\$2) du mot-clé par son sens. \\
A3 & A3 est une fonction lexicale qui, étant donné une lexie comme mot-clé, renvoie une autre lexie adjectivale dérivée qui caractérise le troisième argument (\$3) du mot-clé par son sens. \\
A4 & A4 est une fonction lexicale qui, étant donné une lexie comme mot-clé, renvoie une autre lexie adjectivale dérivée qui caractérise le quatrième argument (\$4) du mot-clé par son sens. \\
A\_dériv\_qualificative & Les fonctions lexicales de dérivations adjectivales qualificatives sont une classe de fonctions lexicales qui, étant donné une lexie comme mot-clé, renvoient un adjectif ou un syntagme adjectival typiquement utilisé pour qualifier l'argument sémantique \$i du mot-clé. Cela exprime soit une capacité ou possibilité, soit une probabilité ou tendance. \\
Able\_i & Les Able\_i sont une classe de fonctions lexicales qui, étant donné une lexie comme mot-clé, renvoient une autre lexie adjectivale dérivée qui caractérise l'argument \$i de ce mot-clé. Ce dérivé exprime une capacité ou une possibilité, souvent formulée par 'tel que \$i peut \textasciitilde ' ou 'tel qu'on peut \textasciitilde \space \$i'. \\
Able1 & Able1 est une fonction lexicale qui, étant donné une lexie comme mot-clé, renvoie une autre lexie adjectivale dérivée qui caractérise le premier argument (\$1) de ce mot-clé. Cette dérivation exprime une capacité ou une possibilité, souvent formulée par 'tel que \$1 peut \textasciitilde '. \\
Able2 & Able2 est une fonction lexicale qui, étant donné une lexie comme mot-clé, renvoie une autre lexie adjectivale dérivée qui caractérise le deuxième argument (\$2) de ce mot-clé. Cette dérivation exprime une capacité ou une possibilité, souvent formulée par 'tel qu'il peut \textasciitilde \space \$2'. \\
Qual\_i & Qual\_i est une classe de fonctions lexicales qui, étant donné une lexie comme mot-clé, renvoient une autre lexie adjectivale dérivée qui caractérise l'argument \$i comme ayant tendance à faire/être \textasciitilde . \\
Qual1 & Qual1 est une fonction lexicale qui, étant donné une lexie comme mot-clé, renvoie une autre lexie adjectivale dérivée qui caractérise le premier argument (\$1) du mot-clé, comme ayant tendance à faire/être \textasciitilde . \\
Qual2 & Qual2 est une fonction lexicale qui, étant donné une lexie comme mot-clé, renvoie une autre lexie adjectivale dérivée qui caractérise le deuxième argument (\$2) du mot-clé, comme ayant tendance à faire/être \textasciitilde . \\
Qual3 & Qual3 est une fonction lexicale qui, étant donné une lexie comme mot-clé, renvoie une autre lexie adjectivale dérivée qui caractérise le troisième argument (\$3) du mot-clé, comme ayant tendance à faire/être \textasciitilde . \\
Dérivation\_adverbiale & Les fonctions lexicales de dérivation adverbiale sont une classe de fonctions lexicales qui, étant donné une lexie comme mot-clé, renvoient une autre lexie adverbiale dérivée tout en préservant un lien sémantique avec ce mot-clé. \\
Adv\_dériv\_sens\_vide & Adv\_dériv\_sens\_vide sont une classe de fonctions lexicales qui, étant donné une lexie comme mot-clé, renvoient une autre lexie adverbiale dérivée de même sens. \\
Adv0 & Adv0 est une fonction lexicale qui, étant donné une lexie comme mot-clé, renvoie une autre lexie adverbiale dérivée de même sens que le mot-clé. \\
Adv\_i & Adv\_i sont une classe de fonctions lexicales qui, étant donné une lexie comme mot-clé, renvoient une autre lexie adverbiale dérivée, typiquement utilisé pour décrire la manière dont l'actant \$i du mot-clé agit ou se manifeste \\
Adv1 & Adv1 est une fonction lexicale qui, étant donné une lexie comme mot-clé, renvoie une autre lexie adverbiale dérivée qui caractérise la manière dont le premier argument (\$1) du mot-clé agit ou se manifeste par le sens du mot-clé. \\
Adv2 & Adv2 est une fonction lexicale qui, étant donné une lexie comme mot-clé, renvoie une autre lexie adverbiale dérivée qui caractérise la manière dont le deuxième argument (\$) du mot-clé agit ou se manifeste par le sens du mot-clé. \\
Dérivation\_verbale & Les fonctions lexicales de dérivation verbale sont une classe de fonctions lexicales qui, étant donné une lexie comme mot-clé, renvoient une autre lexie verbale dérivée dérivé à partir de ce mot-clé. \\
V\_dériv\_sens\_vide & Les fonctions lexicales de dérivations verbales vides de sens sont une classe de fonctions lexicales qui, étant donné une lexie comme mot-clé, renvoient une autre lexie verbale dérivée dérivé à partir du mot-clé sans ajout de sens. \\
V0 & V0 est une fonction lexicale qui, étant donné une lexie comme mot-clé, renvoie une autre lexie verbale dérivée dérivé ayant exactement le même sens que le mot-clé. \\
Pred & Pred est une fonction lexicale qui, étant donné une lexie comme mot-clé, renvoie une autre lexie verbale qui correspond au prédicat sémantique central du mot-clé. \\
V\_dériv\_ajout\_de\_sens & V\_dériv\_ajout\_de\_sens sont une classe de fonctions lexicales qui, étant donné une lexie comme mot-clé, renvoient une lexie verbale sémantiquement liée au mot-clé mais dont le sens n'est pas exactement le même \\
Result\_i & Les Result\_i sont une famille de fonctions lexicales qui, étant donné une lexie comme mot-clé, renvoient une autre lexie verbale qui décrit la situation de l'argument \$i résultant nécessairement du fait de mot-clé.  \\
Result1 & Result1 est une fonction lexicale qui, étant donné une lexie comme mot-clé, renvoie une autre lexie verbale dérivée qui décrit la situation de l'argument \$1 résultant nécessairement du fait de mot-clé. \\
Result2 & Result2 est une fonction lexicale qui, étant donné une lexie comme mot-clé, renvoie une autre lexie verbale dérivée qui décrit la situation du deuxième argument (\$2) résultant nécessairement du fait de mot-clé. \\
FL\_syntagmatiques & Les fonctions lexicales syntagmatiques sont un groupe de fonctions lexicales qui associent un mot-clé à un collocatif fréquent, en fonction de contraintes syntaxiques et sémantiques. \\
Modificateur & Les fonctions lexicales de collocations modificateurs sont une classe de fonctions lexicales qui, étant donné une lexie comme mot-clé, sélectionnent une autre lexie comme collocatif en rôle d'adjectif ou d'adverbe destiné à modifier le mot-clé dans une collocation. \\
Modif\_sens\_vide & Les fonctions lexicales de collocations modificateurs sont une classe de fonctions lexicales qui, étant donné une lexie comme mot-clé, sélectionnent une autre lexie comme collocatif en rôle d'adjectif ou d'adverbe dont le sens est déjà implicite dans la définition du mot-clé \\
Epit & Epit est une fonction lexicale qui, étant donné une lexie comme mot-clé, sélectionne une autre lexie comme un collocatif épithète pléonastique qui modifie le mot-clé dans une collocation qui forme un cliché stylistique. \\
Redun & Redun est une fonction lexicale qui, étant donné une lexie comme mot-clé, sélectionne une autre lexie comme un collocatif épithète redondant qui modifie le mot-clé dans une collocation pour en préciser l'acception. \\
Modif\_ajout\_de\_sens & Les fonctions lexicales de collocations avec ajout de sens sont une classe de fonctions lexicales qui, étant donné une lexie comme mot-clé, sélectionnent une autre lexie comme collocatif en rôle d'adjectif ou d'adverbe qui modifie le mot-clé avec un sens spécifique. \\
Magn & Magn est une fonction lexicale qui, étant donné une lexie comme mot-clé, sélectionne une autre lexie comme collocatif en rôle d'adjectif ou d'adverbe qui modifie le mot-clé en exprimant l'intensité ou la grandeur du mot-clé. \\
Ver & Ver est une fonction lexicale qui, étant donné une lexie comme mot-clé, sélectionne une autre lexie comme collocatif en rôle d'adjectif ou d'adverbe qui modifie le mot-clé en exprimant le sens 'tel qu'il faut', c'est-à-dire une conformité à une norme implicite par rapport au mot-clé. \\
Bon & Bon est une fonction lexicale qui, étant donné une lexie comme mot-clé, sélectionne une autre lexie comme collocatif en rôle d'adjectif ou d'adverbe qui modifie le mot-clé en exprimant un jugement positif de la part du locuteur par rapport au mot-clé. \\
Colloc\_préposition & Les fonctions lexicales de préposition sont une classe de fonctions lexicales qui, étant donné une lexie comme mot-clé, sélectionnent une préposition qui l'introduit dans une collocation. \\
Loc & Loc est une fonction lexicale qui, étant donné une lexie comme mot-clé, sélectionne une autre lexie prépositionelle introduisant le mot-clé en tant que lieu dans une collocation. \\
Instr & Instr est une fonction lexicale qui, étant donné une lexie comme mot-clé, sélectionne une autre lexie prépositionnelle exprimant le moyen ou l'instrument, en l'introduisant dans une collocation signifiant 'au moyen de \textasciitilde '. \\
Propt & Propt est une fonction lexicale qui, étant donné une lexie comme mot-clé, sélectionne une préposition ou un syntagme prépositionnel qui introduit le mot-clé en collocation signifiant 'à cause de \textasciitilde '. \\
Nom\_gouverneur & Les fonctions lexicales de gouverneurs nominaux sont une classe de fonctions lexicales qui, étant donné une lexie comme mot-clé, sélectionnent une autre lexie comme collocatif nominal qui gouverne le mot-clé dans une collocation. \\
Unité & Les fonctions lexicales de gouverneurs nominaux d'unité sont une classe de fonctions lexicales qui, étant donné une lexie comme mot-clé, sélectionne une autre lexie nominale gouvernant le mot-clé dans une collocation qui désigne une unité typique ou un individu responsable de ce mot-clé. \\
Cap & Cap est une fonction lexicale qui, étant donné une lexie comme mot-clé, sélectionne une autre lexie nominale gouvernant le mot-clé dans une collocation qui désigne un individu responsable de ce mot-clé. \\
Sing & Sing est une fonction lexicale qui, étant donné une lexie comme mot-clé, sélectionne une autre lexie nominale gouvernant le mot-clé dans une collocation qui désigne une unité de ce mot-clé. \\
Groupe & Les fonctions lexicale de gouverneurs nominaux de groupe sont une classe de fonctions lexicales qui, étant donné une lexie comme mot-clé, sélectionne une autre lexie nominale gouvernant le mot-clé dans une collocation qui désigne un ensemble régulier ou un groupe effectuant l'activité du mot-clé. \\
Mult & Mult est une fonction lexicale qui, étant donné une lexie comme mot-clé, sélectionne une autre lexie nominale gouvernant le mot-clé dans une collocation qui désigne un ensemble régulier de ce mot-clé. \\
Equip & Equip est une fonction lexicale qui, étant donné une lexie comme mot-clé, sélectionne une autre lexie nominale gouvernant le mot-clé dans une collocation qui désigne un groupe effectuant l'activité du mot-clé. \\
Collocation\_verbale & Les fonctions lexicales de collocations verbales sont une classe de fonctions lexicales qui, étant donné une lexie comme mot-clé, sélectionnent un verbe pour construire une collocation avec ce mot-clé. \\
V\_collocation\_sens\_vide & V\_collocation\_sens\_vide regroupe les fonctions lexicales qui désignent une valeur verbale en collocation du mot-clé mais qui n'apportent pas de sens supplémentaire. \\
Copul & Copul est une fonction lexicale qui, étant donné une lexie comme mot-clé, renvoie comme valeur sa copule typique, qui signifie 'être'. \\
Verbe\_support & Les verbes supports sont une classe de fonctions lexicales qui, étant donné une lexie comme mot-clé, sélectionnent une autre lexie comme collocatif verbal qui sert de support syntaxique sans ajouter de sens. \\
Func\_i & Les Func\_i sont une classe de fonctions lexicales qui, étant donné une lexie comme mot-clé, sélectionnent une autre lexie comme un collocatif pour construire une collocation permettant son insertion dans une phrase sans en modifier le sens. Dans la collocation, le mot-clé occupe la position de sujet du verbe, tandis que son argument sémantique \$i (s'il y a lieu) devient le complément d'objet. \\
Func0 & Func0 est une fonction lexicale qui, étant donné une lexie comme mot-clé, sélectionne une autre lexie comme un collocatif pour construire une collocation permettant son insertion dans une phrase sans en modifier le sens. Dans la collocation, le mot-clé joue le rôle de sujet du verbe sans qu'auncun autre argument du mot-clé se manifeste dans la phrase. \\
Func1 & Func1 est une fonction lexicale qui, étant donné une lexie comme mot-clé, sélectionne une autre lexie comme un collocatif pour construire une collocation permettant son insertion dans une phrase sans en modifier le sens. Dans cette structure, le mot-clé joue le rôle de sujet du verbe, et son premier argument (\$1) devient le complément d'objet. \\
Func2 & Func2 est une fonction lexicale qui, étant donné une lexie comme mot-clé, sélectionne une autre lexie comme un collocatif pour construire une collocation permettant son insertion dans une phrase sans en modifier le sens. Dans cette structure, le mot-clé joue le rôle de sujet du verbe, et son deuxième argument (\$2) devient le complément d'objet. \\
Func3 & Func3 est une fonction lexicale qui , étant donné un mot-clé, sélectionne une autre lexie comme un collocatif pour construire une collocation permettant son insertion dans une phrase sans en modifier le sens. Dans cette structure, le mot-clé joue le rôle de sujet du verbe, et son troisième argument (\$3) devient le complément d'objet. \\
Oper\_i & Les Oper\_i sont une classe de fonctions lexicales qui, étant donné une lexie comme mot-clé, sélectionnent une autre lexie comme un collocatif pour construire une collocation permettant son insertion dans une phrase sans en modifier le sens. Dans la collocation, le mot-clé joue le rôle de complément d'objet direct du verbe, tandis que son argument sémantique \$i devient le sujet syntaxique. \\
Oper1 & Oper1 est une fonction lexicale qui, étant donné une lexie comme mot-clé, sélectionne une autre lexie comme un collocatif pour construire une collocation permettant son insertion dans une phrase sans en modifier le sens. Dans cette structure, le mot-clé joue le rôle de complément d'objet direct du collocatif, et son premier argument (\$1) devient le sujet syntaxique. \\
Oper2 & Oper2 est une fonction lexicale qui, étant donné une lexie comme mot-clé, sélectionne une autre lexie comme collocatif verbal pour construire une collocation permettant son insertion dans une phrase sans en modifier le sens. Dans cette structure, le mot-clé joue le rôle de complément d'objet direct du collocatif, et son deuxième argument (\$2) devient le sujet syntaxique. \\
Oper3 & Oper3 est une fonction lexicale qui, étant donné une lexie comme mot-clé, sélectionne une autre lexie comme collocatif verbal pour construire une collocation permettant son insertion dans une phrase sans en modifier le sens. Dans la collocation, le mot-clé joue le rôle de complément d'objet direct du collocatif, et son troisième argument (\$3) devient le sujet syntaxique. \\
Oper4 & Oper4 est une fonction lexicale qui, étant donné une lexie comme mot-clé, sélectionne une autre lexie comme collocatif verbal pour construire une collocation permettant son insertion dans une phrase sans en modifier le sens. Dans la collocation, le mot-clé joue le rôle de complément d'objet direct du collocatif, et son quatrième argument (\$4) devient le sujet syntaxique. \\
V\_collocation\_ajout\_de\_sens & Les fonctions lexicales de collocations verbales avec ajout de sens sont une classe de fonctions lexicales qui, étant donné une lexie comme mot-clé, sélectionnent une autre lexie comme collocatif verbal pour construire une collocation exprimant un sens spécifique. Dans la collocation, le verbe prend le mot-clé comme sujet ou objet syntaxique. \\
Vreal & Les fonctions lexicales de collocations de verbes de réalisation sont une classe de fonctions lexicales qui, étant donné une lexie comme mot-clé, sélectionnent une autre lexie comme collocatif verbal pour construire une collocation exprimant le sens '\textasciitilde \space se réalise', '\textasciitilde \space fonctionne' ou 'utiliser \textasciitilde '. \\
Real\_i & Les Real\_i sont une classe de fonctions lexicales qui, étant donné une lexie comme mot-clé, sélectionnent une autre lexie comme collocatif verbal pour construire une collocation. Dans cette collocation le verbe prend le mot-clé comme complément d'objet direct, exprimant l'idée de '\$i utilise \textasciitilde ' dont l'argument \$i du mot-clé occupe la position de sujet du verbe. \\
Real1 & Real1 est une fonction lexicale qui, étant donné une lexie comme mot-clé, sélectionne une autre lexie comme collocatif verbal pour construire une collocation. Dans cette collocation, le collocatif verbal prend le mot-clé comme complément d'objet direct en exprimant le sens de '\$1 utilise \textasciitilde ', et le premier argument (\$1) du mot-clé joue un rôle de sujet du verbe. \\
Real2 & Real2 est une fonction lexicale qui, étant donné une lexie comme mot-clé, sélectionne une autre lexie comme collocatif verbal pour construire une collocation. Dans cette collocation, le collocatif verbal prend le mot-clé comme complément d'objet direct en exprimant le sens de '\$2 utilise \textasciitilde ', et le deuxième argument (\$2) du mot-clé joue un rôle de sujet du verbe. \\
Real3 & Real3 est une fonction lexicale qui, étant donné une lexie comme mot-clé, sélectionne une autre lexie comme collocatif verbal pour construire une collocation. Dans cette collocation, le collocatif verbal prend le mot-clé comme complément d'objet direct en exprimant le sens de '\$3 utilise \textasciitilde ', et le troisième argument (\$3) du mot-clé joue un rôle de sujet du verbe. \\
Real4 & Real4 est une fonction lexicale qui, étant donné une lexie comme mot-clé, sélectionne une autre lexie comme collocatif verbal pour construire une collocation. Dans cette collocation, le collocatif verbal prend le mot-clé comme complément d'objet direct en exprimant le sens de '\$4 utilise \textasciitilde ', et le quatrième argument (\$4) du mot-clé joue un rôle de sujet du verbe. \\
RealΩ & RealΩ est une fonction lexicale qui, étant donné une lexie comme mot-clé, sélectionne une autre lexie comme collocatif verbal pour construire une collocation. Dans cette collocation, le collocatif verbal prend le mot-clé comme complément d'objet direct en exprimant le sens de '\$Ω utiliser \textasciitilde ', dont le sujet \$Ω n'est pas un argument sémantique du mot-clé. \\
Fact\_i & Fact\_i sont une classe de fonctions lexicales qui, étant donné une lexie comme mot-clé, sélectionnent une autre lexie comme collocatif verbal pour construire une collocation. Dans cette collocation, le collocatif verbal prend le mot-clé comme sujet et exprime un sens '\textasciitilde \space fonctionne pour \$i' ou '\textasciitilde \space se réalise pour \$i'. \\
Fact0 & Fact0 est une fonction lexicale qui, étant donné une lexie comme mot-clé, sélectionne une autre lexie comme collocatif verbal pour construire une collocation. Dans cette collocation, le collocatif verbal prend le mot-clé comme sujet et exprime un sens '\textasciitilde \space fonctionne' ou '\textasciitilde \space se réalise', et aucun autre argument du mot-clé ne se manifeste comme dépendant syntaxique du verbe. \\
Fact1 & Fact1 est une fonction lexicale qui, étant donné une lexie comme mot-clé, sélectionne une autre lexie comme collocatif verbal pour construire une collocation. Dans cette collocation, le collocatif verbal prend le mot-clé comme sujet et exprime un sens '\textasciitilde \space fonctionne pour \$1' ou '\textasciitilde \space se réalise pour \$1', dont le premier argument (\$1) du mot-clé joue un rôle de complément d'objet du verbe. \\
Fact2 & Fact2 est une fonction lexicale qui, étant donné une lexie comme mot-clé, sélectionne une autre lexie comme collocatif verbal pour construire une collocation. Dans cette collocation, le collocatif verbal prend le mot-clé comme sujet et exprime un sens '\textasciitilde \space fonctionne pour \$2' ou '\textasciitilde \space se réalise pour \$2', dont le deuxième argument (\$2) du mot-clé joue un rôle de complément d'objet du verbe. \\
Labreal\_ij & Labreal\_ij sont une classe de fonctions lexicales qui, étant donné une lexie comme mot-clé, sélectionne une autre lexie comme collocatif verbal pour construire une collocation. Dans cette collocation, le collocatif verbal prend le mot-clé comme complément d'objet indirect et exprime un sens '\$i soumet \$j à l'action de \textasciitilde '. \\
Labreal12 & Labreal12 est une fonction lexicale qui, étant donné une lexie comme mot-clé, sélectionne une autre lexie comme collocatif verbal pour construire une collocation. Dans cette collocation, le collocatif verbal prend le mot-clé comme complément d'objet indirect et exprime un sens '\$1 soumet \$2 à l'action de \textasciitilde '. \\
Labreal13 & Labreal12 est une fonction lexicale qui, étant donné une lexie comme mot-clé, sélectionne une autre lexie comme collocatif verbal pour construire une collocation. Dans cette collocation, le collocatif verbal prend le mot-clé comme complément d'objet indirect et exprime un sens '\$1 soumet \$3 à l'action de \textasciitilde '. \\
Labreal21 & Labreal12 est une fonction lexicale qui, étant donné une lexie comme mot-clé, sélectionne une autre lexie comme collocatif verbal pour construire une collocation. Dans cette collocation, le collocatif verbal prend le mot-clé comme complément d'objet indirect et exprime un sens '\$2 soumet \$1 à l'action de \textasciitilde '. \\
Verbes\_de\_phase & Les fonctions lexicales de collocation Verbes\_de\_phase sont une classe de fonctions lexicales qui, étant donné une lexie comme mot-clé, sélectionnent une autre lexie comme collocatif pour construire une collocation exprimant un changement de phase du mot-clé, soit les sens 'début', 'fin', 'continuation', ou 'préparation'. \\
Incep & Incep est une fonction lexicale qui, étant donné une lexie comme mot-clé, sélectionne une autre lexie comme collocatif verbal pour construire une collocation exprimant l'action de commencement du mot-clé. \\
Cont & Cont est une fonction lexicale qui, étant donné une lexie comme mot-clé, sélectionne une autre lexie comme collocatif verbal pour construire une collocation exprimant la continuation du mot-clé. \\
Fin & Fin est une fonction lexicale qui, étant donné une lexie comme mot-clé, sélectionne une autre lexie comme collocatif verbal pour construire une collocation exprimant la mise en fin du mot-clé. \\
Prepar & Prepar est une fonction lexicale qui, étant donné une lexie comme mot-clé, sélectionne une autre lexie comme collocatif verbal pour construire une collocation exprimant la préparation du mot-clé. \\
Verbes\_causatifs & Les verbes causatifs sont une classe de fonctions lexicales qui, pour un mot-clé donné, renvoient un collocatif verbal de ce mot-clé qui exprime un sens lié à la causalité: 'causer \textasciitilde ', 'faire cesser \textasciitilde ' ou 'ne pas faire cesser \textasciitilde '. \\
Caus & Caus est une fonction lexicale qui, étant donné une lexie comme mot-clé, sélectionne une autre lexie comme collocatif verbal pour construire une collocation exprimant généralement le sens 'causer que \textasciitilde \space ait lieu'. \\
Perm & Perm est une fonction lexicale qui, étant donné une lexie comme mot-clé, sélectionne une autre lexie comme collocatif verbal pour construire une collocation exprimant généralement le sens 'ne pas faire cesser \textasciitilde '. \\
Liqu & Liqu est une fonction lexicale qui, étant donné une lexie comme mot-clé, sélectionne une autre lexie comme collocatif verbal pour construire une collocation exprimant le sens 'causer que \textasciitilde \space s'arrête' ou 'faire cesser \textasciitilde '. \\
Verbes\_sens\_d'action & Verbes\_sens\_d'action est une classe de fonctions lexicales qui prennent un mot-clé en argument et renvoient un collocatif verbal de ce mot-clé, en ajoutant un sens supplémentaire. Le verbe prend le mot-clé comme sujet et exprime un sens '\textasciitilde \space produit un son' ou '\textasciitilde \space se manifeste sur…', etc. \\
Involv & Involv est une fonction lexicale qui, étant donné une lexie comme mot-clé, sélectionne une autre lexie comme collocatif verbal pour construire une collocation. Le verbe prend le mot-clé comme premier actant syntaxique (sujet) dans la phrase et exprime le sens '\textasciitilde \space agit sur qqch \\
Manif & Manif est une fonction lexicale qui, étant donné une lexie comme mot-clé, sélectionne une autre lexie comme collocatif verbal pour construire une collocation. Le verbe prend le mot-clé comme premier actant syntaxique (sujet) dans la phrase et exprime le sens '\textasciitilde \space se manifeste dans qqch'. \\
Son & Son est une fonction lexicale qui, étant donné une lexie comme mot-clé, sélectionne une autre lexie comme collocatif verbal pour construire une collocation. Le verbe prend le mot-clé comme premier actant syntaxique (sujet) dans la phrase et exprime le sens '\textasciitilde \space produit un son typique'. \\
\end{longtable}
